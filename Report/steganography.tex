\chapter{Steganography}

The steganography is a technique that allows us to hide a message into an image, of every kind of format, of course to be effective this technique has to be used on the RAW image, so before doing the compression. 

One of the simplest adopted techniques in the literature involves the changing of at most the 3 least significant bits for every bytes that represents the color of a pixel.  
In some experiments they proved that if we use only the least significant bit, the message is practically undetectable because the human eye isn't so sensitive. 
Other techniques exploit the trasformations that need to be done in the compression phase, but in this way the image appears a little more changed.
Steganography may be used both with or without cryptography, so it's a kind of an external mechanism that allows the parties to communicate in an undetected way. 

Steganography is easily detected by applying lossy compression algorithms, in fact with this technique we try to exploit the strong correlations that there are between adjacent pixels, of course the change of a bit reduces this correlation, so the modifications introduced with steganography can be detected when we apply one of this types of algorithms.  